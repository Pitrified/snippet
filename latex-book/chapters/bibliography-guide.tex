% https://it.overleaf.com/learn/latex/Bibliography_management_in_LaTeX

\section{Bibliography}

Using \texttt{biblatex} you can display bibliography divided into sections, 
depending of citation type. 
Let's cite! Einstein's journal paper \cite{einstein} and the Dirac's 
book \cite{dirac} are physics related items. 
Next, \textit{The \LaTeX\ Companion} book \cite{latexcompanion}, the Donald 
Knuth's website \cite{knuthwebsite}, \textit{The Comprehensive Tex Archive 
Network} (CTAN) \cite{ctan} are \LaTeX\ related items; but the others Donald 
Knuth's items \cite{knuth-fa,knuth-acp} are dedicated to programming. 

% https://www.overleaf.com/learn/latex/Hyperlinks
\section{Hyperlinks and urls}

\begin{equation}
\label{eq:1}
\sum_{i=0}^{\infty} a_i x^i
\end{equation}
 
The equation \ref{eq:1} shows a sum that is divergent. This formula 
will later be used in the page \pageref{sss:esempio}.
 
For further references see \href{http://www.sharelatex.com}{Something 
Linky} or go to the next url: \url{http://www.sharelatex.com} or open 
the bibliography \href{run:./chapters/bibliography-guide.tex}{source file}.
Opening files is a shaky operation.
 
It's also possible to link directly any word or 
\hyperlink{ht:thesentence}{any sentence} in your document.

If you read this text, you will get noinformation.  Really?  Is there no information?

For instance \hypertarget{ht:thesentence}{this sentence} actually has a \texttt{hypertarget} embedded in it.
 
\subsection{Sottosezione}
\label{sss:esempio}

Questa è una sottosezione d'esempio, con un label, a una certa pagina.
