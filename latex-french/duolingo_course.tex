\documentclass[a4paper,11pt,oneside]{book}
\usepackage[T1]{fontenc}
\usepackage[utf8]{inputenc}
\usepackage[english]{babel}

\usepackage{lipsum}

\usepackage{enumitem} % more control over lists: stackoverflow.com/a/4974583
% \usepackage{indentfirst} % indent after section/chapter heading
\usepackage{multirow}

% do not include subsection in in the toc
\setcounter{tocdepth}{1}

\usepackage{hyperref}	% enable hyperlinks to referenced elements
\hypersetup{
	colorlinks=true,
	linkcolor=cyan,
	citecolor=cyan,
    linkcolor=cyan,
    filecolor=magenta,
    urlcolor=blue,
}
\urlstyle{same}

% magic headers
\usepackage{fancyhdr}
\makeatletter
\pagestyle{fancy}
\lhead{\@title}
\chead{}
\rhead{}
% https://tex.stackexchange.com/a/340126
% \lfoot{\if@mainmatter Chapter \thechapter\fi}
% \lfoot{\if@mainmatter Section \thesection\fi}
\lfoot{\if@mainmatter \rightmark\fi}
\cfoot{\@author}
\rfoot{\thepage}
\makeatother 

\title{French lessons}
\author{Pietro}
\date{February 2020}


%%%%% COMMANDS %%%%%

%% phrases %%
\newcommand{\phrase}[2]{\noindent\textbf{#1}\\*\-\hspace{0.5cm}\textit{#2}\\}

%% verbs %%

\newcommand{\verbconj}[8]{
\begin{tabular}{rl}
#1 & \MakeUppercase{#2} \\ 
je & \textbf{#3} \\  
tu & \textbf{#4} \\
il/elle & \textbf{#5} \\  
nous & \textbf{#6} \\  
vous & \textbf{#7} \\  
ils/elles & \textbf{#8}  
\end{tabular}
}

\newcommand{\verbconjcol}[7]{
\begin{tabular}{l}
\MakeUppercase{#1} \\ 
\textbf{#2} \\  
\textbf{#3} \\
\textbf{#4} \\  
\textbf{#5} \\  
\textbf{#6} \\  
\textbf{#7}  
\end{tabular}
}

%% nouns %%

% declension with MFxSP
\newcommand{\noundecl}[4]{
\begin{tabular}{rcc}
 & masculine & feminine \\
 sing. & \textbf{#1} & \textbf{#2} \\
 plur. & \textbf{#3} & \textbf{#4}
\end{tabular}
}

% declension with MFxSP and translations
\newcommand{\noundecltr}[8]{
\begin{tabular}{rcc}
 & masculine & feminine \\
 \multirow{2}{*}{sing.} & \textbf{#1} & \textbf{#2} \\
& \textit{#3} & \textit{#4} \\
 \multirow{2}{*}{plur.} & \textbf{#5} & \textbf{#6} \\
& \textit{#7} & \textit{#8}
\end{tabular}
}

% definition
\newcommand{\noundef}[4]{\textbf{#1} (#3) (pl. #2): \textit{#4}\\}
% \newcommand{\noundef}[4]{#3 \textbf{#1} (pl. #2): \textit{#4}}


% sections
% \newcommand{\lexicon}{\subsection{Lexicon}}
\newcommand{\lexicon}{\subsection{Vocabulary}}
\newcommand{\phrases}{\subsection{Phrases}}
\newcommand{\nouns}{\subsubsection{Nouns}}
\newcommand{\adjectives}{\subsubsection{Adjectives}}

\newcommand{\outref}[1]{\url{#1}}

\begin{document}
 
\frontmatter

\maketitle

\tableofcontents

\mainmatter

\chapter{Grammar and vocabulary}

\section{Grammar}

\subsection{Adjectives}

French adjectives usually come after the noun they’re describing.

\subsection{Preposition}

\subsubsection{Space}

French has several words for in.
Use \textbf{à} with cities but \textbf{en} with many countries
(more info on countries and cities: \outref{https://forum.wordreference.com/threads/countries-pays.229586/}).

% TODO
a le
a au aux


\textbf{En} is used for transportation.
Je vais au Canada \textbf{en} avion; Je voyage \textbf{en} Europe \textbf{en} train.
\textbf{En} is also used for time when you're saying how long it takes to do something.
Je peux construire une maison \textbf{en} 6 mois.
It is also used for months, years, and seasons (except printemps which uses au).

\textbf{Dans} is used when you are literally in something.
Tu peux être \textbf{dans} le bus, \textbf{dans} l'avion, \textbf{dans} un bar, \textbf{dans} une maison.
% Les choses et les gens peuvent être dans un état.
\textbf{Dans} is also used for time when you're saying when something in the future will happen.
J'arrive \textbf{dans} un quart d'heure (I'll be there in 15 minutes).


\section{Time}

\subsubsection{Days}

\subsubsection{Months}

\subsubsection{Seasons}

\section{Pronouns}

\subsection{Possessive}
Possessive adjectives (\textit{les adjectifs possessifs}) in French are used to signify that one person or thing belongs to another and are usually placed in front of the noun they refer to. 
French possessive adjectives change depending on the gender of the noun they're describing.

\begin{center}
\noundecl{mon}{ma}{mes}{mes}
\noundecl{ton}{ta}{tes}{tes}

\noundecl{son}{sa}{ses}{ses}
\noundecl{notre}{notre}{nos}{nos}

\noundecl{votre}{votre}{vos}{vos}
\noundecl{leur}{leur}{leurs}{leurs}
\end{center}

Note that \textbf{mon}, \textbf{ton} and \textbf{son} are used before feminine nouns or adjectives beginning with a vowel or silent h.

Possessive pronouns can substitute a noun (\textit{le pronom possessif}), that has generally been mentioned before, and change according to the noun they substitute.

\phrase{Ce n’est pas mon chapeau, c’est \emph{le sien}.}{It's not my hat, it's hers.}


\noundecl{le mien}{la mienne}{les miens}{les miennes}
\noundecl{le tien}{la tienne}{les tiens}{les tiennes}

\noundecl{le sien}{la sienne}{les siens}{les siennes}
\noundecl{le nôtre}{la nôtre}{les nôtres}{les nôtres}

\noundecl{le vôtre}{la vôtre}{les vôtres}{les vôtres}
\noundecl{le leur}{la leur}{les leurs}{les leurs}

\section{Nouns}

\subsubsection{Countries}

spain
england

\section{Special chars}

à
â
æ
è
é
ê
ë
î
ï
ô
ù
û
ü
ç
œ


\chapter{First castle}

\section{Basics 1}

\noundef{chat}{chats}{m}{cat}
\noundef{cheval}{chevaux}{m}{horse}
\noundef{chien}{chiens}{m}{dog}
\noundef{croissant}{croissants}{m}{croissant}
\noundef{femme}{femmes}{f}{woman, wife}
\noundef{fille}{filles}{f}{girl, daughter}
\noundef{garçon}{garçons}{m}{boy, young man, waiter}
\noundef{homme}{hommes}{m}{man}
\noundef{orange}{oranges}{f}{orange}
\noundef{pizza}{pizzas}{f}{pizza}

\phrase{C'est un chien.}{It's a dog.}
\phrase{Tu es Alice?}{Are you Alice?}
\phrase{Tu es un chat.}{You are a cat.}
\phrase{Je suis un homme.}{I am a man.}
\phrase{Paul mange un croissant.}{Paul is eating a croissant.}
\phrase{Tu es un cheval?}{Are you a horse?}
\phrase{Un garçon mange un croissant.}{A boy is eating a croissant.}

\begin{center}
\verbconj{Présent}{être}{suis}{es}{est}{sommes}{êtes}{sont}
\verbconjcol{manger}{mange}{manges}{mange}{mangeons}{mangez}{mangent}
\end{center}

\section{Greetings}

\phrase{Au revoir, bonne journée!}{Goodbye, have a good day!}
\phrase{Au revoir, bonne soirée.}{Goodbye, have a good evening.}
\phrase{Au revoir, à bientôt!}{Goodbye, see you soon!}
\phrase{Bienvenue.}{Welcome.}
\phrase{Bonjour Marie, enchanté.}{Good morning Marie, nice to meet you.}
\phrase{Bonjour.}{Good day.}
\phrase{Bonne nuit, Paul.}{Good night, Paul.}
\phrase{Bonsoir, comment ça va?}{Good evening, how are you doing?}
\phrase{Bonsoir.}{Good evening.}
\phrase{Oui, et toi?}{Yes, and you?}
\phrase{Oui, merci.}{Yes, thank you.}
\phrase{Oui, ça va bien.}{Yes, I am doing well.}
\phrase{Salut, bonne nuit!}{Bye, goodnight!}
\phrase{Salut, bonne soirée.}{Bye, have a good evening.}
\phrase{Salut, comment ça va?}{Hi, how are you?}
\phrase{Salut, ça va?}{Hi, how are you?}
\phrase{Salut.}{Hello.}
\phrase{À demain!}{See you tomorrow!}
\phrase{Ça va, et toi?}{I’m fine, and you?}

\section{Basics 2}

\begin{center}
\noundecl{américain}{américaine}{américains}{américaines}
\noundecl{anglais}{anglaise}{anglais}{anglaises}

\noundecl{espagnol}{espagnole}{espagnols}{espagnoles}
\noundecl{français}{française}{français}{françaises}

\noundecl{mexicain}{mexicaine}{mexicains}{mexicaines}
\end{center}

\phrase{Duo est américain?}{Is Duo American?}
\phrase{Excuse-moi, comment tu t'appelles?}{Excuse me, what is your name?}
\phrase{Excuse-moi, tu parles français?}{Excuse me, do you speak French?}
\phrase{Il parle anglais.}{He speaks English.}
\phrase{Je m'appelle Julia.}{My name is Julia.}
\phrase{Julia est mexicaine.}{Julia is Mexican.}
\phrase{Marc est américain.}{Marc is American.}
\phrase{Paul parle français.}{Paul speaks French.}

\begin{center}
\verbconj{Présent}{parler}{parle}{parles}{parle}{parlons}{parlez}{parlent}
\end{center}

\section{People}

\noundef{hôpital}{hôpitaux}{m}{hospital}
\noundef{journaliste}{journalistes}{m/f}{journalist, anchorman or anchorwoman, reporter}

\phrase{J’habite à Paris.}{I live in Paris.}
\phrase{Je travaille à Madrid. }{I work in Madrid.}
\phrase{Je suis à Londres.}{I’m in London.}
\phrase{J’habite en France.}{I live in France.}
\phrase{Je travaille en Espagne.}{I work in Spain.}
\phrase{Je suis en Angleterre.}{I’m in England.}
\phrase{Elle est étudiante.}{She is a student.}
\phrase{Non, merci.}{No, thank you.}
\phrase{Oui, en Angleterre.}{Yes, in England.}
\phrase{Paul est journaliste.}{Paul is a journalist.}
\phrase{Tu habites à Bordeaux?}{Do you live in Bordeaux?}

\begin{center}
\verbconj{Présent}{travailler}{travaille}{travailles}{travaille}{travaillons}{travaillez}{travaillent}
\verbconjcol{étudier}{étudie}{étudies}{étudie}{étudions}{étudiez}{étudient}
\verbconjcol{habiter}{habite}{habites}{habite}{habitons}{habitez}{habitent}
\end{center}

\section{Travel}

\noundef{passeport}{passeports}{m}{passport}
\noundef{taxi}{taxis}{m}{taxi}
\noundef{voiture}{voitures}{f}{car}
\noundef{hôtel}{hôtels}{m}{hotel}
\noundef{avion}{avions}{m}{plane}
\noundef{billet}{billets}{m}{ticket, note, banknote}
\noundef{restaurant}{restaurants}{m}{restaurant}
\noundef{train}{trains}{m}{train}
\noundef{aéroport}{aéroports}{m}{airport}

\phrase{A croissant, please.}{Un croissant, s'il vous plaît.}
\phrase{C'est un avion français.}{It is a French plane.}
\phrase{C'est une valise.}{It is a suitcase.}
\phrase{C'est une valise.}{It is a suitcase?}
\phrase{Elle a un billet d'avion.}{She has a plane ticket.}
\phrase{Elle va à l'hôtel.}{She is going to the hotel.}
\phrase{Il va à la gare.}{He is going to the train station.}
\phrase{Je prends la voiture.}{I am taking the car.}
\phrase{Je vais à Montréal.}{I am going to Montreal.}
\phrase{Où est la gare, s'il vous plaît ?}{Where is the train station, please?}
\phrase{Tu as un billet d'avion ?}{Do you have a plane ticket?}
\phrase{Tu prends l'avion?}{Are you taking the plane?}
\phrase{Tu prends une valise ?}{Are you taking a suitcase?}
\phrase{Tu vas à la gare en taxi ?}{Are you going to the train station by taxi?}

\begin{center}
\verbconj{Présent}{aller}{vais}{vas}{va}{allons}{allez}{vont}
\verbconjcol{avoir}{ai}{as}{a}{avons}{avez}{ont}
\end{center}

\section{Family}

\noundef{animal de compagnie}{animaux de compagnie}{m}{pet}
\noundef{chouette}{chouettes}{f}{owl}
\noundef{famille}{familles}{f}{family}
\noundef{fille}{filles}{f}{daughter}
\noundef{fils}{fils}{m}{son}
\noundef{frère}{frères}{m}{brother}
\noundef{grand-mère}{grand-mères}{f}{grandmother}
\noundef{grand-père}{grand-pères}{m}{grandfather}
\noundef{jardin}{jardins}{m}{garden}
\noundef{maison}{maisons}{f}{house}
\noundef{mari}{maris}{m}{husband}
\noundef{mère}{mères}{f}{mother}
\noundef{père}{pères}{m}{father}
\noundef{sœur}{sœurs}{f}{sister}

\phrase{Duo est une chouette.}{Duo is an owl.}
\phrase{J'ai une chouette.}{I have an owl.}
\phrase{Ma famille habite en France.}{My family lives in France.}
\phrase{Ma fille est française.}{My daughter is French.}
\phrase{Marc habite avec sa mère.}{Marc lives with his mother.}
\phrase{Mon fils habite en France.}{My son lives in France.}
\phrase{Paul veut un animal de compagnie.}{Paul wants a pet.}
\phrase{Sa grand-mère est mexicaine.}{His grandmother is Mexican.}
\phrase{Ton frère est américain.}{Your brother is American.}
\phrase{Ton mari va bien ?}{Is your husband doing well?}
\phrase{Ton père va bien ?}{Is your father doing well?}
\phrase{Tu as un animal de compagnie ?}{Do you have a pet?}
\phrase{Tu vas très bien.}{You are doing very well.}
\phrase{Une maison avec un jardin.}{A house with a garden.}
\phrase{Vous avez une fille?}{Do you have a daughter?}

\begin{center}
\verbconj{Présent}{vouloir}{veux}{veux}{veut}{voulons}{voulez}{veulent}
\end{center}

\section{Activities}

\noundef{banque}{banques}{f}{bank}
\noundef{boulangerie}{boulangeries}{f}{bakery}
\noundef{bus}{bus}{m/f}{bus}
\noundef{chocolat}{chocolats}{m}{chocolate}
\noundef{livre}{livres}{m}{book}
\noundef{musique}{musiques}{f}{music}
\noundef{métro}{métros}{m}{subway, underground}
\noundef{zoo}{zoos}{m}{zoo}
\noundef{école}{écoles}{f}{school}


\phrase{Je vais au travail en métro.}{I go to work by subway.}
\phrase{Je vais à la banque en voiture.}{I am going to the bank by car.}
\phrase{Nous aimons Duo.}{We like Duo.}
\phrase{Nous aimons lire.}{We like to read.}
\phrase{Nous allons à l'hôtel.}{We are going to the hotel.}
\phrase{Nous allons à la boulangerie.}{We are going to the bakery.}
\phrase{Nous avons un chat.}{We have a cat.}
\phrase{Nous habitons en France.}{We live in France.}
\phrase{Nous habitons en France.}{We live in France.}
\phrase{Nous mangeons un croissant.}{We are eating a croissant.}
\phrase{Nous sommes à la banque.}{We are at the bank.}
\phrase{Nous voulons lire un livre.}{We want to read a book.}
\phrase{Nous voulons une pizza.}{We want a pizza.}
\phrase{Tu prends le métro ?}{Do you take the subway?}
\phrase{Tu veux un chat ?}{Do you want a cat?}
\phrase{Tu veux une orange ?}{Do you want an orange?}

\begin{center}
\verbconj{Présent}{lire}{lis}{lis}{lit}{lisons}{lisez}{lisent}
\end{center}

\section{People 2}

\noundef{professeur}{professeurs}{m}{teacher, professor}

\phrase{Elles ont deux voitures.}{They have two cars.}
\phrase{Elles parlent anglais.}{They speak English.}
\phrase{Il a trois chats.}{He has three cats.}
\phrase{Ils mangent deux pizzas.}{They eat two pizzas. / They are eating two pizzas.}
\phrase{Ils ont un chat.}{They have a cat.}
\phrase{Ils étudient le français.}{They study French.}
\phrase{Je mange beaucoup.}{I eat a lot.}
\phrase{Les filles sont étudiantes.}{The girls are students.}
\phrase{Les professeurs habitent ici.}{The professors live here.}

\begin{center}
\verbconj{Présent}{prendre}{prends}{prends}{prend}{prenons}{prenez}{prennent}
\end{center}

\section{Family 2}

\noundef{content}
\noundef{intelligent}
amusant

\phrase{Ce sont mes filles.}{These are my daughters.}
\phrase{Ce sont tes filles ?}{Are these your daughters?}
\phrase{Elle a un fils.}{She has a son.}
\phrase{Elles sont françaises.}{They are French.}
\phrase{Elles sont mexicaines.}{They are Mexican.}
\phrase{Il est content.}{He is happy.}
\phrase{Ils sont mexicains.}{They are Mexican.}
\phrase{J'ai deux fils.}{I have two sons.}
\phrase{Ma fille est amusante.}{My daughter is funny.}
\phrase{Marc est intelligent.}{Marc is smart.}
\phrase{Mes filles sont anglaises.}{My daughters are English.}
\phrase{Mes sœurs sont amusantes.}{My sisters are funny.}
\phrase{Mon chien est intelligent.}{My dog is intelligent.}
\phrase{Mon frère est content.}{My brother is happy.}


\begin{center}
\verbconj{Présent}{aimer}{aime}{aimes}{aime}{aimons}{aimez}{aiment}
\verbconj{Présent}
\end{center}

\section{Restaurant}



\chapter{Second castle}

\section{City}

\lexicon

\noundef{hôpital}{hôpitaux}{m}{hospital}
\noundef{pharmacie}{pharmacies}{f}{pharmacy, drugstore}
\noundef{plante}{plantes}{f}{plant}
\noundef{supermarché}{supermarchés}{m}{supermarket}
\noundef{vert}{verts}{m}{green}
\noundef{ville}{villes}{f}{city}

\noundecl{vert}{verte}{verts}{vertes}

\phrases

\phrase{Il est au supermarché.}{He is at the supermarket.}
\phrase{Il mange de la salade verte.}{He is eating green salad.}
\phrase{Je suis dans la pharmacie.}{I am in the pharmacy.}
\phrase{La maison est petite.}{The house is small.}
\phrase{Le cinéma est ouvert.}{The movie theater is open.}
\phrase{Le magasin est fermé.}{The store is closed.}
\phrase{Les universités sont grandes.}{The universities are big.}
\phrase{Les voitures sont grandes.}{The cars are big.}
\phrase{Les écoles sont grandes.}{The schools are big.}
\phrase{Où est l'hôpital?}{Where is the hospital?}

\section{Travel 2}
\section{At Home}
\section{At Work}
\section{Food}
\section{Habits}
\section{Shopping}
\section{People 3}
\section{City 2}
\section{Friends}
\section{People 4}
\section{At Home 2}
\section{Travel 3}
\section{Activities}
\section{Breakfast}
\section{Vacation}
\section{School}
\section{At Work 2}
\section{Hotel}
\section{Routine}
\section{Weather}
\section{People 5}

\chapter{Third castle}

\section{Sensations}
\section{Groceries}

\phrase{Ils préfèrent ce supermarché.}{They prefer this supermarket.}
\phrase{Ils achètent des tomates et des œufs.}{They are buying tomatoes and eggs.}
\phrase{Elle préfère les œufs.}{She prefers eggs.}
\phrase{Elles préfèrent les tomates vertes.}{They prefer green tomatoes.}
\phrase{Je n'aime pas faire les courses.}{I don't like to go grocery shopping.}
\phrase{Je dois acheter du café.}{I have to buy some coffee.}
\phrase{Il préfère faire les courses au supermarché.}{He prefers to go grocery shopping at the supermarket.}
\phrase{Tu dois acheter du sucre.}{You have to buy some sugar.}
\phrase{Ils préfèrent des tomates vertes.}{They prefer green tomatoes.}




\section{Shopping 2}
\section{City 3}
\section{Routine 2}
\section{Leisure}
\section{Opinion}
\section{Friends 2}
\section{Nature}
\section{Family 3}
\section{School 2}
\section{Food 2}
\section{Routine 3}
\section{Travel 4}
\section{Health}
\section{Housing}
\section{At Work 3}
\section{Memories}
\section{Leisure 2}
\section{At Home 3}
\section{Travel 5}


\end{document}

% idiomax.com copia bene i verbi
% coniugazione.reverso.net indica il verbo di modello
