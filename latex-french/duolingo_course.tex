\documentclass[a4paper,11pt,oneside]{book}
\usepackage[T1]{fontenc}
\usepackage[utf8]{inputenc}
\usepackage[english]{babel}

\usepackage{lipsum}

\usepackage{enumitem} % more control over lists: stackoverflow.com/a/4974583
% \usepackage{indentfirst} % indent after section/chapter heading
\usepackage{multirow}

% do not include subsection in in the toc
\setcounter{tocdepth}{1}

\usepackage{hyperref}	% enable hyperlinks to referenced elements
\hypersetup{
	colorlinks=true,
	linkcolor=cyan,
	citecolor=cyan,
    linkcolor=cyan,
    filecolor=magenta,
    urlcolor=blue,
}
\urlstyle{same}

% magic headers
\usepackage{fancyhdr}
\makeatletter
\pagestyle{fancy}
\lhead{\@title}
\chead{}
\rhead{}
% https://tex.stackexchange.com/a/340126
% \lfoot{\if@mainmatter Chapter \thechapter\fi}
% \lfoot{\if@mainmatter Section \thesection\fi}
\lfoot{\if@mainmatter \rightmark\fi}
\cfoot{\@author}
\rfoot{\thepage}
\makeatother 

\title{French lessons}
\author{Pietro}
\date{February 2020}


%%%%% COMMANDS %%%%%

%% phrases %%
\newcommand{\phrase}[2]{\noindent\textbf{#1}\\*\-\hspace{0.5cm}\textit{#2}\\}

%% verbs %%

\newcommand{\verbconj}[8]{
\begin{tabular}{rl}
#1 & \MakeUppercase{#2} \\ 
je & \textbf{#3} \\  
tu & \textbf{#4} \\
il/elle & \textbf{#5} \\  
nous & \textbf{#6} \\  
vous & \textbf{#7} \\  
ils/elles & \textbf{#8}  
\end{tabular}
}

\newcommand{\verbconjcol}[7]{
\begin{tabular}{l}
\MakeUppercase{#1} \\ 
\textbf{#2} \\  
\textbf{#3} \\
\textbf{#4} \\  
\textbf{#5} \\  
\textbf{#6} \\  
\textbf{#7}  
\end{tabular}
}

%% nouns %%

% declension with MFxSP
\newcommand{\noundecl}[4]{
\begin{tabular}{rcc}
 & masculine & feminine \\
 sing. & \textbf{#1} & \textbf{#2} \\
 plur. & \textbf{#3} & \textbf{#4}
\end{tabular}
}

% declension with MFxSP and translations
\newcommand{\noundecltr}[8]{
\begin{tabular}{rcc}
 & masculine & feminine \\
 \multirow{2}{*}{sing.} & \textbf{#1} & \textbf{#2} \\
& \textit{#3} & \textit{#4} \\
 \multirow{2}{*}{plur.} & \textbf{#5} & \textbf{#6} \\
& \textit{#7} & \textit{#8}
\end{tabular}
}

% definition
\newcommand{\noundef}[4]{\textbf{#1} (#3) (pl. #2): \textit{#4}\\}
% \newcommand{\noundef}[4]{#3 \textbf{#1} (pl. #2): \textit{#4}}


% sections
% \newcommand{\lexicon}{\subsection{Lexicon}}
\newcommand{\lexicon}{\subsection{Vocabulary}}
\newcommand{\phrases}{\subsection{Phrases}}
\newcommand{\nouns}{\subsubsection{Nouns}}
\newcommand{\adjectives}{\subsubsection{Adjectives}}

% idiomax.com copia bene i verbi
% coniugazione.reverso.net indica il verbo di modello

\begin{document}
 
\frontmatter

\maketitle

\tableofcontents

\mainmatter

\chapter{Grammar and vocabulary}

\section{Grammar}

\subsection{Adjectives}

French adjectives usually come after the noun they’re describing.

\subsection{Preposition}

\subsubsection{Space}

French has two words for in. Use \textbf{à} with cities but \textbf{en} with many countries.

\section{Time}

\subsubsection{Days}

\subsubsection{Months}

\subsubsection{Seasons}

\section{Nouns}

\subsubsection{Countries}

% MAYBE in a group on top
spain
england

\chapter{First castle}

\section{Basics 1}

\noundef{chat}{chats}{m}{cat}
\noundef{cheval}{chevaux}{m}{horse}
\noundef{chien}{chiens}{m}{dog}
\noundef{croissant}{croissants}{m}{croissant}
\noundef{femme}{femmes}{f}{woman, wife}
\noundef{fille}{filles}{f}{girl, daughter}
\noundef{garçon}{garçons}{m}{boy, young man, waiter}
\noundef{homme}{hommes}{m}{man}
\noundef{orange}{oranges}{f}{orange}
\noundef{pizza}{pizzas}{f}{pizza}

\phrase{C'est un chien.}{It's a dog.}
\phrase{Tu es Alice?}{Are you Alice?}
\phrase{Tu es un chat.}{You are a cat.}
\phrase{Je suis un homme.}{I am a man.}
\phrase{Paul mange un croissant.}{Paul is eating a croissant.}
\phrase{Tu es un cheval?}{Are you a horse?}
\phrase{Un garçon mange un croissant.}{A boy is eating a croissant.}

\begin{center}
\verbconj{Présent}{être}{suis}{es}{est}{sommes}{êtes}{sont}
\verbconjcol{manger}{mange}{manges}{mange}{mangeons}{mangez}{mangent}
\end{center}

\section{Greetings}

\phrase{Au revoir, bonne journée!}{Goodbye, have a good day!}
\phrase{Au revoir, bonne soirée.}{Goodbye, have a good evening.}
\phrase{Au revoir, à bientôt!}{Goodbye, see you soon!}
\phrase{Bienvenue.}{Welcome.}
\phrase{Bonjour Marie, enchanté.}{Good morning Marie, nice to meet you.}
\phrase{Bonjour.}{Good day.}
\phrase{Bonne nuit, Paul.}{Good night, Paul.}
\phrase{Bonsoir, comment ça va?}{Good evening, how are you doing?}
\phrase{Bonsoir.}{Good evening.}
\phrase{Oui, et toi?}{Yes, and you?}
\phrase{Oui, merci.}{Yes, thank you.}
\phrase{Oui, ça va bien.}{Yes, I am doing well.}
\phrase{Salut, bonne nuit!}{Bye, goodnight!}
\phrase{Salut, bonne soirée.}{Bye, have a good evening.}
\phrase{Salut, comment ça va?}{Hi, how are you?}
\phrase{Salut, ça va?}{Hi, how are you?}
\phrase{Salut.}{Hello.}
\phrase{À demain!}{See you tomorrow!}
\phrase{Ça va, et toi?}{I’m fine, and you?}

\section{Basics 2}

\begin{center}
\noundecl{américain}{américaine}{américains}{américaines}
\noundecl{anglais}{anglaise}{anglais}{anglaises}

\noundecl{espagnol}{espagnole}{espagnols}{espagnoles}
\noundecl{français}{française}{français}{françaises}

\noundecl{mexicain}{mexicaine}{mexicains}{mexicaines}
\end{center}

\phrase{Duo est américain?}{Is Duo American?}
\phrase{Excuse-moi, comment tu t'appelles?}{Excuse me, what is your name?}
\phrase{Excuse-moi, tu parles français?}{Excuse me, do you speak French?}
\phrase{Il parle anglais.}{He speaks English.}
\phrase{Je m'appelle Julia.}{My name is Julia.}
\phrase{Julia est mexicaine.}{Julia is Mexican.}
\phrase{Marc est américain.}{Marc is American.}
\phrase{Paul parle français.}{Paul speaks French.}

\begin{center}
\verbconj{Présent}{parler}{parle}{parles}{parle}{parlons}{parlez}{parlent}
\end{center}

\section{People}

\noundef{hôpital}{hôpitaux}{m}{hospital}
\noundef{journaliste}{journalistes}{m/f}{journalist, anchorman or anchorwoman, reporter}

\phrase{J’habite à Paris.}{I live in Paris.}
\phrase{Je travaille à Madrid. }{I work in Madrid.}
\phrase{Je suis à Londres.}{I’m in London.}
\phrase{J’habite en France.}{I live in France.}
\phrase{Je travaille en Espagne.}{I work in Spain.}
\phrase{Je suis en Angleterre.}{I’m in England.}
\phrase{Elle est étudiante.}{She is a student.}
\phrase{Non, merci.}{No, thank you.}
\phrase{Oui, en Angleterre.}{Yes, in England.}
\phrase{Paul est journaliste.}{Paul is a journalist.}
\phrase{Tu habites à Bordeaux?}{Do you live in Bordeaux?}

\begin{center}
\verbconj{Présent}{travailler}{travaille}{travailles}{travaille}{travaillons}{travaillez}{travaillent}
\verbconjcol{étudier}{étudie}{étudies}{étudie}{étudions}{étudiez}{étudient}
\verbconjcol{habiter}{habite}{habites}{habite}{habitons}{habitez}{habitent}
\end{center}

\section{Travel}

\noundef{passeport}{passeports}{m}{passport}
\noundef{taxi}{taxis}{m}{taxi}
\noundef{voiture}{voitures}{f}{car}
\noundef{hôtel}{hôtels}{m}{hotel}
\noundef{avion}{avions}{m}{plane}
\noundef{billet}{billets}{m}{ticket, note, banknote}
\noundef{restaurant}{restaurants}{m}{restaurant}
\noundef{train}{trains}{m}{train}
\noundef{aéroport}{aéroports}{m}{airport}

\phrase{A croissant, please.}{Un croissant, s'il vous plaît.}
\phrase{C'est un avion français.}{It is a French plane.}
\phrase{C'est une valise.}{It is a suitcase.}
\phrase{C'est une valise.}{It is a suitcase?}
\phrase{Elle a un billet d'avion.}{She has a plane ticket.}
\phrase{Elle va à l'hôtel.}{She is going to the hotel.}
\phrase{Il va à la gare.}{He is going to the train station.}
\phrase{Je prends la voiture.}{I am taking the car.}
\phrase{Je vais à Montréal.}{I am going to Montreal.}
\phrase{Où est la gare, s'il vous plaît ?}{Where is the train station, please?}
\phrase{Tu as un billet d'avion ?}{Do you have a plane ticket?}
\phrase{Tu prends l'avion?}{Are you taking the plane?}
\phrase{Tu prends une valise ?}{Are you taking a suitcase?}
\phrase{Tu vas à la gare en taxi ?}{Are you going to the train station by taxi?}

\begin{center}
\verbconj{Présent}{aller}{vais}{vas}{va}{allons}{allez}{vont}
\verbconjcol{avoir}{ai}{as}{a}{avons}{avez}{ont}
\end{center}

\section{Family}
\section{Activities}
\section{People 2}
\section{Family 2}
\section{Restaurant}

\chapter{Second castle}

\section{City}

\lexicon

\noundef{hôpital}{hôpitaux}{m}{hospital}
\noundef{pharmacie}{pharmacies}{f}{pharmacy, drugstore}
\noundef{plante}{plantes}{f}{plant}
\noundef{supermarché}{supermarchés}{m}{supermarket}
\noundef{vert}{verts}{m}{green}
\noundef{ville}{villes}{f}{city}

\noundecl{vert}{verte}{verts}{vertes}

\phrases

\phrase{Il est au supermarché.}{He is at the supermarket.}
\phrase{Il mange de la salade verte.}{He is eating green salad.}
\phrase{Je suis dans la pharmacie.}{I am in the pharmacy.}
\phrase{La maison est petite.}{The house is small.}
\phrase{Le cinéma est ouvert.}{The movie theater is open.}
\phrase{Le magasin est fermé.}{The store is closed.}
\phrase{Les universités sont grandes.}{The universities are big.}
\phrase{Les voitures sont grandes.}{The cars are big.}
\phrase{Les écoles sont grandes.}{The schools are big.}
\phrase{Où est l'hôpital?}{Where is the hospital?}

\section{Travel 2}
\section{At Home}
\section{At Work}
\section{Food}
\section{Habits}
\section{Shopping}
\section{People 3}
\section{City 2}
\section{Friends}
\section{People 4}
\section{At Home 2}
\section{Travel 3}
\section{Activities}
\section{Breakfast}
\section{Vacation}
\section{School}
\section{At Work 2}
\section{Hotel}
\section{Routine}
\section{Weather}
\section{People 5}

\chapter{Third castle}

\section{Sensations}
\section{Groceries}
\section{Shopping 2}
\section{City 3}
\section{Routine 2}
\section{Leisure}
\section{Opinion}
\section{Friends 2}
\section{Nature}
\section{Family 3}
\section{School 2}
\section{Food 2}
\section{Routine 3}
\section{Travel 4}
\section{Health}
\section{Housing}
\section{At Work 3}
\section{Memories}
\section{Leisure 2}
\section{At Home 3}
\section{Travel 5}


\end{document}

