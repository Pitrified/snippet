\documentclass[a4paper,11pt,oneside]{book}
\usepackage[T1]{fontenc}
\usepackage[utf8]{inputenc}
\usepackage[english]{babel}

\usepackage{lipsum}

\usepackage{enumitem} % more control over lists: stackoverflow.com/a/4974583
% \usepackage{indentfirst} % indent after section/chapter heading
\usepackage{multirow}

% do not include subsection in in the toc
\setcounter{tocdepth}{1}

\usepackage{hyperref}	% enable hyperlinks to referenced elements
\hypersetup{
	colorlinks=true,
	linkcolor=cyan,
	citecolor=cyan,
    linkcolor=cyan,
    filecolor=magenta,
    urlcolor=blue,
}
\urlstyle{same}

% magic headers
\usepackage{fancyhdr}
\makeatletter
\pagestyle{fancy}
\lhead{\@title}
\chead{}
\rhead{}
% https://tex.stackexchange.com/a/340126
% \lfoot{\if@mainmatter Chapter \thechapter\fi}
% \lfoot{\if@mainmatter Section \thesection\fi}
\lfoot{\if@mainmatter \rightmark\fi}
\cfoot{\@author}
\rfoot{\thepage}
\makeatother 

\title{French lessons}
\author{Pietro}
\date{February 2020}


%%%%% COMMANDS %%%%%

%%%% phrases %%%%

\newcommand{\phrase}[2]{\noindent\textbf{#1}\\*\-\hspace{0.5cm}\textit{#2}\\}

%%%% verbs %%%%

\newcommand{\verbconj}[8]{
    \begin{tabular}{rl}
    #1 & \MakeUppercase{#2} \\ 
    je & \textbf{#3} \\  
    tu & \textbf{#4} \\
    il/elle & \textbf{#5} \\  
    nous & \textbf{#6} \\  
    vous & \textbf{#7} \\  
    ils/elles & \textbf{#8}  
    \end{tabular}
}

\newcommand{\verbconjm}[9]{
\begin{tabular}{rl}
#1 & \MakeUppercase{#2} \\ 
\textit{model} & \textit{#9} \\  
je & \textbf{#3} \\  
tu & \textbf{#4} \\
il/elle & \textbf{#5} \\  
nous & \textbf{#6} \\  
vous & \textbf{#7} \\  
ils/elles & \textbf{#8}  
\end{tabular}
}

\newcommand{\verbconjcol}[7]{
\begin{tabular}{l}
\MakeUppercase{#1} \\ 
\textbf{#2} \\  
\textbf{#3} \\
\textbf{#4} \\  
\textbf{#5} \\  
\textbf{#6} \\  
\textbf{#7}  
\end{tabular}
}

\newcommand{\verbconjcolm}[8]{
\begin{tabular}{l}
\MakeUppercase{#1} \\ 
\textit{#8} \\  
\textbf{#2} \\  
\textbf{#3} \\
\textbf{#4} \\  
\textbf{#5} \\  
\textbf{#6} \\  
\textbf{#7}  
\end{tabular}
}

%%%% nouns/adjectives %%%%

% declension with MFxSP
\newcommand{\noundecl}[4]{
\begin{tabular}{rcc}
 & masculine & feminine \\
 sing. & \textbf{#1} & \textbf{#2} \\
 plur. & \textbf{#3} & \textbf{#4}
\end{tabular}
}

% declension with MFxSP and translations
\newcommand{\noundecltr}[8]{
\begin{tabular}{rcc}
 & masculine & feminine \\
 \multirow{2}{*}{sing.} & \textbf{#1} & \textbf{#2} \\
& \textit{#3} & \textit{#4} \\
 \multirow{2}{*}{plur.} & \textbf{#5} & \textbf{#6} \\
& \textit{#7} & \textit{#8}
\end{tabular}
}

% definition
\newcommand{\noundef}[4]{\textbf{#1} (#3) (pl. #2): \textit{#4}\\}
% \newcommand{\noundef}[4]{#3 \textbf{#1} (pl. #2): \textit{#4}}


% sections
% \newcommand{\lexicon}{\subsection{Lexicon}}
\newcommand{\lexicon}{\subsection{Vocabulary}}
\newcommand{\phrases}{\subsection{Phrases}}
\newcommand{\nouns}{\subsubsection{Nouns}}
\newcommand{\adjectives}{\subsubsection{Adjectives}}

% styling links
\newcommand{\outref}[1]{\url{#1}}

% french phrases in esamples
\newcommand{\frtext}[1]{\emph{#1}}


\begin{document}
 
\frontmatter

\maketitle

\tableofcontents

\mainmatter

\chapter{Grammar and vocabulary}

\section{Grammar}

\subsection{Adjectives}

French adjectives usually come after the noun they’re describing.

\subsection{Preposition}

\subsubsection{Space}

French has several words for in.
Use \textbf{à} with cities but \textbf{en} with many countries
(more info on countries and cities: \outref{https://forum.wordreference.com/threads/countries-pays.229586/}).

% TODO
TODO a le a au aux


\frtext{\textbf{En}} is used for transportation.
\frtext{Je vais au Canada \textbf{en} avion};
\frtext{Je voyage \textbf{en} Europe \textbf{en} train.}
\frtext{\textbf{En}} is also used for time when you're saying how long it takes to do something.
\frtext{Je peux construire une maison \textbf{en} 6 mois.}
It is also used for months, years, and seasons (except \frtext{printemps} which uses \frtext{\textbf{au}}).

\frtext{\textbf{Dans}} is used when you are literally in something.
\frtext{Tu peux être \textbf{dans} le bus, \textbf{dans} l'avion, \textbf{dans} un bar, \textbf{dans} une maison.}
% Les choses et les gens peuvent être dans un état.
\frtext{\textbf{Dans}} is also used for time when you're saying when something in the future will happen.
\frtext{J'arrive \textbf{dans} un quart d'heure (I'll be there in 15 minutes).}


\section{Time}

\subsubsection{Days}

\subsubsection{Months}

\subsubsection{Seasons}

\section{Pronouns}

\subsection{Possessive}
Possessive adjectives (\textit{les adjectifs possessifs}) in French are used to signify that one person or thing belongs to another and are usually placed in front of the noun they refer to. 
French possessive adjectives change depending on the gender of the noun they're describing.

\begin{center}
\noundecl{mon}{ma}{mes}{mes}
\noundecl{ton}{ta}{tes}{tes}

\noundecl{son}{sa}{ses}{ses}
\noundecl{notre}{notre}{nos}{nos}

\noundecl{votre}{votre}{vos}{vos}
\noundecl{leur}{leur}{leurs}{leurs}
\end{center}

Note that \textbf{mon}, \textbf{ton} and \textbf{son} are used before feminine nouns or adjectives beginning with a vowel or silent h.

Possessive pronouns can substitute a noun (\textit{le pronom possessif}), that has generally been mentioned before, and change according to the noun they substitute.

\phrase{Ce n’est pas mon chapeau, c’est \emph{le sien}.}{It's not my hat, it's hers.}


\noundecl{le mien}{la mienne}{les miens}{les miennes}
\noundecl{le tien}{la tienne}{les tiens}{les tiennes}

\noundecl{le sien}{la sienne}{les siens}{les siennes}
\noundecl{le nôtre}{la nôtre}{les nôtres}{les nôtres}

\noundecl{le vôtre}{la vôtre}{les vôtres}{les vôtres}
\noundecl{le leur}{la leur}{les leurs}{les leurs}

\subsection{Demonstrative adjectives}

Demonstrative adjectives are used to indicate a specific noun or nouns (this, that, these, those). In French, they must agree with the noun(s) in number and sometimes gender.
They are used in place of an article, and placed in front of a noun or an adjective + noun.
If used as demonstrative adjective + noun they can be replaced with a demonstrative pronoun.

\begin{center}
\noundecl{ce}{cette}{ces}{ces}
\end{center}

When a singular demonstrative adjective precedes a masculine noun or adjective that begins with a vowel or h \frtext{muet}, \frtext{cet} is used to avoid a hiatus.

French demonstrative adjectives make no distinction between "this" and "that" \frtext{ce}, \frtext{cet}, and \frtext{cette} can each mean either one.
Likewise, \frtext{ces} can mean "these" or "those."
A suffix can be attached to the noun to specify whether the object is nearby (\frtext{-ci}) or far away (\frtext{-là}):
\frtext{cette chaise-ci}, this chair, \frtext{cette chaise-là}, that chair.

\outref{https://www.lawlessfrench.com/grammar/demonstrative-adjectives/}

\section{Nouns}

\subsubsection{Countries}

spain
england

\subsubsection{Colors}

TODO

blanc
vert


\section{Special chars}

à
â
æ
è
é
ê
ë
î
ï
ô
ù
û
ü
ç
œ


\chapter{First castle}

\section{Basics 1}

\noundef{chat}{chats}{m}{cat}
\noundef{cheval}{chevaux}{m}{horse}
\noundef{chien}{chiens}{m}{dog}
\noundef{croissant}{croissants}{m}{croissant}
\noundef{femme}{femmes}{f}{woman, wife}
\noundef{fille}{filles}{f}{girl, daughter}
\noundef{garçon}{garçons}{m}{boy, young man, waiter}
\noundef{homme}{hommes}{m}{man}
\noundef{orange}{oranges}{f}{orange}
\noundef{pizza}{pizzas}{f}{pizza}

\phrase{C'est un chien.}{It's a dog.}
\phrase{Tu es Alice?}{Are you Alice?}
\phrase{Tu es un chat.}{You are a cat.}
\phrase{Je suis un homme.}{I am a man.}
\phrase{Paul mange un croissant.}{Paul is eating a croissant.}
\phrase{Tu es un cheval?}{Are you a horse?}
\phrase{Un garçon mange un croissant.}{A boy is eating a croissant.}

\begin{center}
\verbconjm{Présent}{être}{suis}{es}{est}{sommes}{êtes}{sont}{être}
\verbconjcolm{manger}{mange}{manges}{mange}{mangeons}{mangez}{mangent}{manger}
\end{center}

\section{Greetings}

\phrase{Au revoir, bonne journée!}{Goodbye, have a good day!}
\phrase{Au revoir, bonne soirée.}{Goodbye, have a good evening.}
\phrase{Au revoir, à bientôt!}{Goodbye, see you soon!}
\phrase{Bienvenue.}{Welcome.}
\phrase{Bonjour Marie, enchanté.}{Good morning Marie, nice to meet you.}
\phrase{Bonjour.}{Good day.}
\phrase{Bonne nuit, Paul.}{Good night, Paul.}
\phrase{Bonsoir, comment ça va?}{Good evening, how are you doing?}
\phrase{Bonsoir.}{Good evening.}
\phrase{Oui, et toi?}{Yes, and you?}
\phrase{Oui, merci.}{Yes, thank you.}
\phrase{Oui, ça va bien.}{Yes, I am doing well.}
\phrase{Salut, bonne nuit!}{Bye, goodnight!}
\phrase{Salut, bonne soirée.}{Bye, have a good evening.}
\phrase{Salut, comment ça va?}{Hi, how are you?}
\phrase{Salut, ça va?}{Hi, how are you?}
\phrase{Salut.}{Hello.}
\phrase{À demain!}{See you tomorrow!}
\phrase{Ça va, et toi?}{I’m fine, and you?}

\section{Basics 2}

\begin{center}
\noundecl{américain}{américaine}{américains}{américaines}
\noundecl{anglais}{anglaise}{anglais}{anglaises}

\noundecl{espagnol}{espagnole}{espagnols}{espagnoles}
\noundecl{français}{française}{français}{françaises}

\noundecl{mexicain}{mexicaine}{mexicains}{mexicaines}
\end{center}

\phrase{Duo est américain?}{Is Duo American?}
\phrase{Excuse-moi, comment tu t'appelles?}{Excuse me, what is your name?}
\phrase{Excuse-moi, tu parles français?}{Excuse me, do you speak French?}
\phrase{Il parle anglais.}{He speaks English.}
\phrase{Je m'appelle Julia.}{My name is Julia.}
\phrase{Julia est mexicaine.}{Julia is Mexican.}
\phrase{Marc est américain.}{Marc is American.}
\phrase{Paul parle français.}{Paul speaks French.}

\begin{center}
\verbconjm{Présent}{parler}{parle}{parles}{parle}{parlons}{parlez}{parlent}{marcher}
\end{center}

\section{People}

\noundef{hôpital}{hôpitaux}{m}{hospital}
\noundef{journaliste}{journalistes}{m/f}{journalist, anchorman or anchorwoman, reporter}

\phrase{J’habite à Paris.}{I live in Paris.}
\phrase{Je travaille à Madrid. }{I work in Madrid.}
\phrase{Je suis à Londres.}{I’m in London.}
\phrase{J’habite en France.}{I live in France.}
\phrase{Je travaille en Espagne.}{I work in Spain.}
\phrase{Je suis en Angleterre.}{I’m in England.}
\phrase{Elle est étudiante.}{She is a student.}
\phrase{Non, merci.}{No, thank you.}
\phrase{Oui, en Angleterre.}{Yes, in England.}
\phrase{Paul est journaliste.}{Paul is a journalist.}
\phrase{Tu habites à Bordeaux?}{Do you live in Bordeaux?}

\begin{center}
\verbconjm{Présent}{travailler}{travaille}{travailles}{travaille}{travaillons}{travaillez}{travaillent}{marcher}
\verbconjcolm{étudier}{étudie}{étudies}{étudie}{étudions}{étudiez}{étudient}{marcher}
\verbconjcolm{habiter}{habite}{habites}{habite}{habitons}{habitez}{habitent}{marcher}
\end{center}

\section{Travel}

\noundef{passeport}{passeports}{m}{passport}
\noundef{taxi}{taxis}{m}{taxi}
\noundef{voiture}{voitures}{f}{car}
\noundef{hôtel}{hôtels}{m}{hotel}
\noundef{avion}{avions}{m}{plane}
\noundef{billet}{billets}{m}{ticket, note, banknote}
\noundef{restaurant}{restaurants}{m}{restaurant}
\noundef{train}{trains}{m}{train}
\noundef{aéroport}{aéroports}{m}{airport}

\phrase{A croissant, please.}{Un croissant, s'il vous plaît.}
\phrase{C'est un avion français.}{It is a French plane.}
\phrase{C'est une valise.}{It is a suitcase.}
\phrase{C'est une valise.}{It is a suitcase?}
\phrase{Elle a un billet d'avion.}{She has a plane ticket.}
\phrase{Elle va à l'hôtel.}{She is going to the hotel.}
\phrase{Il va à la gare.}{He is going to the train station.}
\phrase{Je prends la voiture.}{I am taking the car.}
\phrase{Je vais à Montréal.}{I am going to Montreal.}
\phrase{Où est la gare, s'il vous plaît?}{Where is the train station, please?}
\phrase{Tu as un billet d'avion?}{Do you have a plane ticket?}
\phrase{Tu prends l'avion?}{Are you taking the plane?}
\phrase{Tu prends une valise?}{Are you taking a suitcase?}
\phrase{Tu vas à la gare en taxi?}{Are you going to the train station by taxi?}

\begin{center}
\verbconjm{Présent}{aller}{vais}{vas}{va}{allons}{allez}{vont}{aller}
\verbconjcolm{avoir}{ai}{as}{a}{avons}{avez}{ont}{avoir}
\end{center}

\section{Family}

\noundef{animal de compagnie}{animaux de compagnie}{m}{pet}
\noundef{chouette}{chouettes}{f}{owl}
\noundef{famille}{familles}{f}{family}
\noundef{fille}{filles}{f}{daughter}
\noundef{fils}{fils}{m}{son}
\noundef{frère}{frères}{m}{brother}
\noundef{grand-mère}{grand-mères}{f}{grandmother}
\noundef{grand-père}{grand-pères}{m}{grandfather}
\noundef{jardin}{jardins}{m}{garden}
\noundef{maison}{maisons}{f}{house}
\noundef{mari}{maris}{m}{husband}
\noundef{mère}{mères}{f}{mother}
\noundef{père}{pères}{m}{father}
\noundef{sœur}{sœurs}{f}{sister}

\phrase{Duo est une chouette.}{Duo is an owl.}
\phrase{J'ai une chouette.}{I have an owl.}
\phrase{Ma famille habite en France.}{My family lives in France.}
\phrase{Ma fille est française.}{My daughter is French.}
\phrase{Marc habite avec sa mère.}{Marc lives with his mother.}
\phrase{Mon fils habite en France.}{My son lives in France.}
\phrase{Paul veut un animal de compagnie.}{Paul wants a pet.}
\phrase{Sa grand-mère est mexicaine.}{His grandmother is Mexican.}
\phrase{Ton frère est américain.}{Your brother is American.}
\phrase{Ton mari va bien?}{Is your husband doing well?}
\phrase{Ton père va bien?}{Is your father doing well?}
\phrase{Tu as un animal de compagnie?}{Do you have a pet?}
\phrase{Tu vas très bien.}{You are doing very well.}
\phrase{Une maison avec un jardin.}{A house with a garden.}
\phrase{Vous avez une fille?}{Do you have a daughter?}

\begin{center}
\verbconjm{Présent}{vouloir}{veux}{veux}{veut}{voulons}{voulez}{veulent}{vouloir}
\end{center}

\section{Activities}

\noundef{banque}{banques}{f}{bank}
\noundef{boulangerie}{boulangeries}{f}{bakery}
\noundef{bus}{bus}{m/f}{bus}
\noundef{chocolat}{chocolats}{m}{chocolate}
\noundef{livre}{livres}{m}{book}
\noundef{musique}{musiques}{f}{music}
\noundef{métro}{métros}{m}{subway, underground}
\noundef{zoo}{zoos}{m}{zoo}
\noundef{école}{écoles}{f}{school}

\phrase{Je vais au travail en métro.}{I go to work by subway.}
\phrase{Je vais à la banque en voiture.}{I am going to the bank by car.}
\phrase{Nous aimons Duo.}{We like Duo.}
\phrase{Nous aimons lire.}{We like to read.}
\phrase{Nous allons à l'hôtel.}{We are going to the hotel.}
\phrase{Nous allons à la boulangerie.}{We are going to the bakery.}
\phrase{Nous avons un chat.}{We have a cat.}
\phrase{Nous habitons en France.}{We live in France.}
\phrase{Nous habitons en France.}{We live in France.}
\phrase{Nous mangeons un croissant.}{We are eating a croissant.}
\phrase{Nous sommes à la banque.}{We are at the bank.}
\phrase{Nous voulons lire un livre.}{We want to read a book.}
\phrase{Nous voulons une pizza.}{We want a pizza.}
\phrase{Tu prends le métro?}{Do you take the subway?}
\phrase{Tu veux un chat?}{Do you want a cat?}
\phrase{Tu veux une orange?}{Do you want an orange?}

\begin{center}
\verbconjm{Présent}{lire}{lis}{lis}{lit}{lisons}{lisez}{lisent}{lire}
\end{center}

\section{People 2}

\noundef{professeur}{professeurs}{m}{teacher, professor}

\phrase{Elles ont deux voitures.}{They have two cars.}
\phrase{Elles parlent anglais.}{They speak English.}
\phrase{Il a trois chats.}{He has three cats.}
\phrase{Ils mangent deux pizzas.}{They eat two pizzas. / They are eating two pizzas.}
\phrase{Ils ont un chat.}{They have a cat.}
\phrase{Ils étudient le français.}{They study French.}
\phrase{Je mange beaucoup.}{I eat a lot.}
\phrase{Les filles sont étudiantes.}{The girls are students.}
\phrase{Les professeurs habitent ici.}{The professors live here.}

\begin{center}
\verbconjm{Présent}{prendre}{prends}{prends}{prend}{prenons}{prenez}{prennent}{prendre}
\end{center}

\section{Family 2}

\begin{center}
\noundecl{content}{contente}{contents}{contentes}
% content, satisfied, pleased
\noundecl{intelligent}{intelligente}{intelligents}{intelligentes}

\noundecl{amusant}{amusante}{amusants}{amusantes}
\end{center}

\phrase{Ce sont mes filles.}{These are my daughters.}
\phrase{Ce sont tes filles?}{Are these your daughters?}
\phrase{Elle a un fils.}{She has a son.}
\phrase{Elles sont françaises.}{They are French.}
\phrase{Elles sont mexicaines.}{They are Mexican.}
\phrase{Il est content.}{He is happy.}
\phrase{Ils sont mexicains.}{They are Mexican.}
\phrase{J'ai deux fils.}{I have two sons.}
\phrase{Ma fille est amusante.}{My daughter is funny.}
\phrase{Marc est intelligent.}{Marc is smart.}
\phrase{Mes filles sont anglaises.}{My daughters are English.}
\phrase{Mes sœurs sont amusantes.}{My sisters are funny.}
\phrase{Mon chien est intelligent.}{My dog is intelligent.}
\phrase{Mon frère est content.}{My brother is happy.}


\begin{center}
\verbconjm{Présent}{aimer}{aime}{aimes}{aime}{aimons}{aimez}{aiment}{marcher}
\end{center}

\section{Restaurant}

\noundef{tasse}{tasses}{f}{cup}
\noundef{sandwich}{sandwichs}{m}{sandwich}
\noundef{bouteille}{bouteilles}{f}{bottle}
\noundef{verre}{verres}{m}{(usually uncountable): glass (substance), (countable): glass (drinking vessel)}
\noundef{fromage}{fromages}{m}{cheese}
\noundef{serveur}{serveurs}{m, feminine serveuse}{waiter}
\noundef{eau}{eaux}{f}{water}
\noundef{bière}{bières}{f}{beer}
\noundef{pain}{pains}{m}{bread}
\noundef{toilettes}{only}{f}{toilet, lavatory}
\noundef{salade}{salades}{f}{salad}
\noundef{jus}{jus}{m}{juice}
\noundef{addition}{additions}{f}{bill, check, sum}
\noundef{personne}{personnes}{f}{person}
\noundef{vin}{vins}{m}{wine}
\noundef{thé}{thés}{m}{tea}

\phrase{Je voudrais un thé.}{I would like a tea.}
\phrase{L'addition, s'il vous plaît.}{The check, please.}
\phrase{La serveuse est française.}{The waitress is French.}
\phrase{Les toilettes pour femmes.}{The women's restroom.}
\phrase{Nous aimons les restaurants français.}{We like French restaurants.}
\phrase{Paul prend un café.}{Paul is having coffee.}
\phrase{Tu prends un thé?}{Are you having tea?}
\phrase{Une table pour deux personnes, s'il vous plaît.}{A table for two people, please.}
\phrase{Une table pour une personne.}{A table for one person.}

\chapter{Second castle}

\section{City}

\noundef{hôpital}{hôpitaux}{m}{hospital}
\noundef{pharmacie}{pharmacies}{f}{pharmacy, drugstore}
\noundef{plante}{plantes}{f}{plant}
\noundef{supermarché}{supermarchés}{m}{supermarket}
\noundef{vert}{verts}{m}{green}
\noundef{ville}{villes}{f}{city}
\noundef{magasin}{magasins}{m}{shop, store}

\begin{center}
\noundecl{vert}{verte}{verts}{vertes}
\end{center}

\phrase{Il est au supermarché.}{He is at the supermarket.}
\phrase{Il mange de la salade verte.}{He is eating green salad.}
\phrase{Je suis dans la pharmacie.}{I am in the pharmacy.}
\phrase{La maison est petite.}{The house is small.}
\phrase{Le cinéma est ouvert.}{The movie theater is open.}
\phrase{Le magasin est fermé.}{The store is closed.}
\phrase{Les universités sont grandes.}{The universities are big.}
\phrase{Les voitures sont grandes.}{The cars are big.}
\phrase{Les écoles sont grandes.}{The schools are big.}
\phrase{Où est l'hôpital?}{Where is the hospital?}

\section{Travel 2}

\noundef{pays}{pays}{m}{country}
\noundef{château}{châteaux}{m}{castle}
\noundef{bébé}{bébés}{m}{baby}
\noundef{musée}{musées}{m}{museum}
\noundef{église}{églises}{f}{church}
\noundef{plage}{plages}{f}{beach}
\noundef{vacances}{only}{f}{holidays}


\phrase{Elle va aux toilettes.}{She is going to the toilets. / She is going to the restroom.}
\phrase{Je visite le Mexique.}{I am visiting Mexico.}
\phrase{Je visite une église.}{I am visiting a church.}
\phrase{Je voyage en avion.}{I am traveling by plane.}
\phrase{Le Brésil est un pays.}{Brazil is a country.}
\phrase{Le japon est un pays.}{Japan is a country.}
\phrase{Nous allons au musée.}{We are going to the museum.}
\phrase{Sophie visite New York avec son fils.}{Sophie is visiting New York with her son.}
\phrase{Tu aimes voyager?}{Do you like to travel?}
\phrase{Tu vas aux États-Unis?}{Are you going to the United States?}
\phrase{Tu visites Paris?}{Are you visiting Paris?}
\phrase{Tu visites les États-Unis?}{Are you visiting the United States?}
\phrase{Tu voyages avec ta sœur?}{Are you traveling with your sister?}
\phrase{Tu voyages beaucoup?}{Do you travel a lot?}
\phrase{Elle voyage avec Paul.}{She is traveling with Paul.}
\phrase{Elles vont au Brésil.}{They are going to Brazil.}
\phrase{Ils aiment les vacances.}{They like vacations.}
\phrase{Ils veulent une pizza et de la salade.}{They want a pizza and salad.}
\phrase{Ils vont aux toilettes.}{They are going to the restroom.}
\phrase{Ils voyagent.}{They are traveling.}
\phrase{J'aime la plage.}{I like the beach.}
\phrase{Je vais aux toilettes.}{I'm going to the toilets.}
\phrase{Je vais aux États-Unis.}{I am going to the United States.}
\phrase{Nous voyageons en Angleterre.}{We are traveling to England.}

\begin{center}
\verbconjm{Présent}{visiter}{voyage}{voyages}{voyage}{voyageons}{voyagez}{voyagent}{marcher}
\end{center}

\section{At Home}

\noundef{pièce}{pièces}{f}{room}
\noundef{vache}{vaches}{f}{cow}
\noundef{chaise}{chaises}{f}{chair, seat}
\noundef{voisin}{voisins}{m, feminine voisine}{neighbor}
\noundef{arbre}{arbres}{m}{tree}
\noundef{table}{tables}{f}{table (furniture)}
\noundef{fenêtre}{fenêtres}{f}{window}

\begin{center}
\noundecl{gentil}{gentille}{gentils}{gentilles}
\noundecl{aimable}{aimable}{aimables}{aimables}

\noundecl{sympa}{sympa}{sympas}{sympas}
\end{center}

\phrase{Mes voisines sont gentilles.}{My neighbors are kind.}
\phrase{J'ouvre la porte de la voiture.}{I am opening the car door.}
\phrase{Julia et Marie sont voisines.}{Julia and Marie are neighbors.}
\phrase{J'aime les arbres.}{I like trees.}
\phrase{Ma voisine est gentille.}{My neighbor is kind.}
\phrase{Le bus est blanc.}{The bus is white.}
\phrase{Les chaises sont vertes.}{The chairs are green.}
\phrase{Il y a trois arbres.}{There are three trees.}
\phrase{Elles ouvrent les fenêtres.}{They are opening the windows.}

\begin{center}
\verbconjm{Présent}{ouvrir}{ouvre}{ouvres}{ouvre}{ouvrons}{ouvrez}{ouvrent}{ouvrir}
\end{center}

\section{At Work}

\noundef{chauffeur}{chauffeurs}{m, feminine chauffeuse}{driver}
\noundef{lettre}{lettres}{f}{letter}
\noundef{message}{messages}{m}{message}
\noundef{médecin}{médecins}{m}{medic, doctor}
\noundef{métier}{métiers}{m}{job, profession, occupation}
\noundef{ordinateur}{ordinateurs}{m}{computer}
\noundef{portable}{portables}{m}{cell phone}
\noundef{stylo}{stylos}{m}{pen}
\noundef{téléphone}{téléphones}{m}{telephone}
\noundef{vélo}{vélos}{m}{bike, bicycle}

\phrase{C'est mon métier.}{It's my occupation.}
\phrase{Elle écrit un e-mail.}{She is writing an email.}
\phrase{Elle écrit un livre.}{She is writing a book.}
\phrase{Il utilise beaucoup son téléphone.}{He uses his phone a lot.}
\phrase{J'utilise mon portable.}{I am using my cell phone.}
\phrase{J'écris beaucoup.}{I write a lot.}
\phrase{Je suis chauffeur.}{I am a driver.}
\phrase{Les lettres sont sur la table.}{The letters are on the table.}
\phrase{Marc écrit avec un stylo.}{Marc writes with a pen.}
\phrase{Tu utilises ton téléphone?}{Are you using your phone?}
\phrase{Tu utilises un ordinateur?}{Are you using a computer?}
\phrase{Tu écris un livre?}{Are you writing a book?}
\phrase{Tu écris un message?}{Are you writing a message?}
\phrase{Tu écris une lettre?}{Are you writing a letter?}

\begin{center}
    \verbconjm{utiliser}{Présent}{utilise}{utilises}{utilise}{utilisons}{utilisez}{utilisent}{marcher}
    \verbconjm{écrire}{Présent}{écris}{écris}{écrit}{écrivons}{écrivez}{écrivent}{écrire}
\end{center}

\section{Food}

\noundef{banane}{bananes}{f}{banana}
\noundef{déjeuner}{déjeuners}{m}{lunch}
\noundef{dîner}{dîners}{m}{dinner}
\noundef{fruit}{fruit}{m}{fruit}
\noundef{légume}{légumes}{m}{vegetable}
\noundef{matin}{matins}{m}{morning}
\noundef{plat}{plats}{m}{dish, course}
\noundef{pomme}{pommes}{f}{apple}
\noundef{œuf}{œufs}{m}{egg}

\phrase{Ces plats sont bons.}{These dishes are good.}
\phrase{Ces plats sont bons.}{These dishes are good.}
\phrase{Il mange un fruit.}{He is eating a fruit.}
\phrase{Ils prennent un café.}{They are having coffee.}
\phrase{Ils prennent un thé.}{They are having tea.}
\phrase{Ils préparent le dîner.}{They are making dinner.}
\phrase{J'aime les pommes.}{I like apples.}
\phrase{John mange un plat américain.}{John is eating an American dish.}
\phrase{La professeur aime les pommes.}{The teacher likes apples.}
\phrase{Tu cuisines beaucoup.}{You cook a lot.}
\phrase{Tu cuisines un plat français ?}{Are you cooking a French dish?}

\begin{center}
    \verbconjm{cuisiner}{Présent}{cuisine}{cuisines}{cuisine}{cuisinons}{cuisinez}{cuisinent}{marcher}
\end{center}

\section{Habits}


% TODO some more templates for invariable adjecties
\noundef{chaque}{invariable}{dude it's invariable}{each, every}
% TODO and more templates for adverbs
\noundef{souvent}{adverb}{adverb}{often}
\noundef{parfois}{adverb}{adverb}{sometimes}
\noundef{après-midi}{après-midi or après-midis}{m or f}{afternoon}
\noundef{après}{preposition}{adverb}{after, afterwards}
\noundef{enfant}{enfants}{m or f}{child}
\noundef{bureau}{bureaux}{m}{office}
\noundef{journal}{journaux}{m}{newspaper, diary}
\noundef{télé}{télés}{f}{TV}
\noundef{nuit}{nuits}{f}{night}
\noundef{lit}{lits}{m}{bed}
\noundef{jour}{jours}{m}{day}
\noundef{petit déjeuner}{petits déjeuners}{m}{breakfast}

\phrase{Après le dîner, je lis le journal.}{After dinner, I read the newspaper.}
\phrase{Aujourd'hui, c'est lundi.}{Today, it's Monday.}
\phrase{Chaque jour, elles regardent la télé.}{Every day, they watch TV.}
\phrase{Chaque jour, les enfants vont à l'école.}{Every day, the children go to school.}
\phrase{Elles regardent la télé.}{They watch TV.}
\phrase{J'ai un chien et un chat.}{I have a dog and a cat.}
\phrase{J'ai un journal français.}{I have a French newspaper.}
\phrase{Je dors dans mon lit.}{I sleep in my bed.}
\phrase{Je dors ici.}{I sleep here.}
\phrase{Je lis chaque dimanche.}{I read every Sunday.}
\phrase{Je lis chaque soir.}{I read every evening.}
\phrase{L'enfant joue avec le chat.}{The child plays with the cat.}
\phrase{La fille écoute la musique sur son lit.}{The girl is listening to music on her bed.}
\phrase{Nous écoutons parfois la radio.}{We sometimes listen to the radio.}
\phrase{Nous écoutons souvent la radio.}{We often listen to the radio.}
\phrase{Je lis beaucoup.}{I read a lot.}

\begin{center}
    \verbconj{dormir}{Présent}{dors}{dors}{dort}{dormons}{dormez}{dorment}{dormir}
    \verbconj{écouter}{Présent}{écoute}{écoutes}{écoute}{écoutons}{écoutez}{écoutent}{marcher}
    \verbconj{jouer}{Présent}{joue}{joues}{joue}{jouons}{jouez}{jouent}{marcher}
\end{center}

\section{Shopping}

\noundef{vêtement}{vêtements}{m}{garment, in plural clotes, clothing}
\noundef{robe}{robes}{f}{dress}
\noundef{pantalon}{pantalons}{m}{trousers, pants}
\noundef{veste}{vestes}{f}{jacket}
\noundef{prix}{prix}{m}{price, prize}
\noundef{chaussure}{chaussures}{f}{shoe}
\noundef{T-shirt}{T-shirts}{m}{T-shirt}
\noundef{sac}{sacs}{f}{bag}
\noundef{jupe}{jupes}{f}{skirt}

\phrase{Ce pantalon est cher.}{These pants are expensive.}
\phrase{Cette robe est chère.}{This dress is expensive.}
\phrase{Combien?}{How much?}
\phrase{J'achète des jupes.}{I am buying some skirts.}
\phrase{Je veux deux robes.}{I want two dresses.}
\phrase{La pizza est chère.}{The pizza is expensive.}
\phrase{Les chaussures coûtent huit euros.}{The shoes cost eight euros.}
\phrase{Les pantalons sont chers.}{The pants are expensive.}
\phrase{Les pizzas sont chères.}{The pizzas are expensive.}
\phrase{Les robes sont rouges.}{The dresses are red.}
\phrase{Les sacs coûtent cinq euros.}{The bags cost five euros.}
\phrase{Les vêtements coûtent neuf euros.}{The clothes cost nine euros.}
\phrase{Mon T-shirt coûte dix euros.}{My T-shirt costs ten euros.}
\phrase{Tu achètes des chaussures ?}{Are you buying shoes?}
\phrase{Tu achètes une veste verte.}{You are buying a green jacket.}
\phrase{Tu as une veste blanche.}{You have a white jacket.}

\begin{center}
    \verbconj{acheter}{Présent}{achète}{achètes}{achète}{achetons}{achetez}{achètent}{acheter}
\end{center}

\section{People 3}

Il n'est pas brun.
He isn't dark-haired.

Son fils a huit ans.
His son is eight years old.

Marie est brune.
Marie is a brunette.
















\section{City 2}
\section{Friends}
\section{People 4}
\section{At Home 2}
\section{Travel 3}
\section{Activities}
\section{Breakfast}
\section{Vacation}
\section{School}
\section{At Work 2}
\section{Hotel}
\section{Routine}
\section{Weather}
\section{People 5}

\chapter{Third castle}

\section{Sensations}
\section{Groceries}

\phrase{Ils préfèrent ce supermarché.}{They prefer this supermarket.}
\phrase{Ils achètent des tomates et des œufs.}{They are buying tomatoes and eggs.}
\phrase{Elle préfère les œufs.}{She prefers eggs.}
\phrase{Elles préfèrent les tomates vertes.}{They prefer green tomatoes.}
\phrase{Je n'aime pas faire les courses.}{I don't like to go grocery shopping.}
\phrase{Je dois acheter du café.}{I have to buy some coffee.}
\phrase{Il préfère faire les courses au supermarché.}{He prefers to go grocery shopping at the supermarket.}
\phrase{Tu dois acheter du sucre.}{You have to buy some sugar.}
\phrase{Ils préfèrent des tomates vertes.}{They prefer green tomatoes.}




\section{Shopping 2}
\section{City 3}
\section{Routine 2}
\section{Leisure}
\section{Opinion}
\section{Friends 2}
\section{Nature}
\section{Family 3}
\section{School 2}
\section{Food 2}
\section{Routine 3}
\section{Travel 4}
\section{Health}
\section{Housing}
\section{At Work 3}
\section{Memories}
\section{Leisure 2}
\section{At Home 3}
\section{Travel 5}


\end{document}

% idiomax.com copia bene i verbi
% coniugazione.reverso.net indica il verbo di modello
