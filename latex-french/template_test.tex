\documentclass[a4paper,11pt,oneside]{book}
\usepackage[T1]{fontenc}
\usepackage[utf8]{inputenc}
\usepackage[english]{babel}

\usepackage{lipsum}

\usepackage{enumitem} % more control over lists: stackoverflow.com/a/4974583
\usepackage{indentfirst} % YES indent after section/chapter heading
\usepackage{multirow}

\usepackage{hyperref}	% enable hyperlinks to referenced elements
\hypersetup{
	colorlinks=true,
	linkcolor=cyan,
	citecolor=cyan,
    linkcolor=cyan,
    filecolor=magenta,
    urlcolor=blue,
}
\urlstyle{same}

% magic headers
\usepackage{fancyhdr}
\makeatletter
\pagestyle{fancy}
\lhead{\@title}
\chead{}
\rhead{}
% https://tex.stackexchange.com/a/340126
% \lfoot{\if@mainmatter Chapter \thechapter\fi}
% \lfoot{\if@mainmatter Section \thesection\fi}
\lfoot{\if@mainmatter \rightmark\fi}
\cfoot{\@author}
\rfoot{\thepage}
\makeatother 

\title{Template for a grammar book}
\author{Pietro}
\date{February 2020}

\newcommand{\phrasea}[2]{\noindent#1\\#2}
\newcommand{\phraseb}[2]{\noindent\textbf{#1}\\#2}
\newcommand{\phrasec}[2]{\noindent\textbf{#1}\\\textit{#2}}
\newcommand{\phrased}[2]{\noindent\textbf{#1}\\\-\hspace{0.5cm}\textit{#2}}
\newcommand{\phrasee}[2]{\noindent\textbf{#1}\\*\-\hspace{0.5cm}\textit{#2}}

\newcommand{\verbconja}[8]{
\begin{tabular}{rl}
#1 & #2 \\ 
je & #3 \\  
tu & #4 \\
il/elle & #5 \\  
nous & #6 \\  
vous & #7 \\  
ils/elles & #8  
\end{tabular}
}
\newcommand{\verbconjb}[8]{
\begin{center}
\begin{tabular}{rl}
#1 & #2 \\ 
je & #3 \\  
tu & #4 \\
il/elle & #5 \\  
nous & #6 \\  
vous & #7 \\  
ils/elles & #8  
\end{tabular}
\end{center}
}
\newcommand{\verbconjc}[8]{
\begin{center}
\begin{tabular}{rl}
#1 & \MakeUppercase{#2} \\ 
je & #3 \\  
tu & #4 \\
il/elle & #5 \\  
nous & #6 \\  
vous & #7 \\  
ils/elles & #8  
\end{tabular}
\end{center}
}
\newcommand{\verbconjd}[8]{
\begin{tabular}{rl}
#1 & \MakeUppercase{#2} \\ 
% #1 & #2 \\ 
je & \textbf{#3} \\  
tu & \textbf{#4} \\
il/elle & \textbf{#5} \\  
nous & \textbf{#6} \\  
vous & \textbf{#7} \\  
ils/elles & \textbf{#8}  
\end{tabular}
}

\newcommand{\noundecla}[4]{
\begin{tabular}{rcc}
 & masculine & feminine \\
sing. & #1 & #2 \\
plur. & #3 & #4
\end{tabular}
}
\newcommand{\noundeclb}[4]{
\begin{center}
\begin{tabular}{rcc}
 & masculine & feminine \\
sing. & #1 & #2 \\
plur. & #3 & #4
\end{tabular}
\end{center}
}
\newcommand{\noundeclc}[4]{
\begin{tabular}{rcc}
 & masculine & feminine \\
 sing. & \textbf{#1} & \textbf{#2} \\
 plur. & \textbf{#3} & \textbf{#4}
\end{tabular}
}

\newcommand{\noundecltra}[8]{
\begin{tabular}{rcc}
 & masculine & feminine \\
sing. & #1 & #2 \\
& #3 & #4 \\
plur. & #5 & #6 \\
& #7 & #8
\end{tabular}
}
\newcommand{\noundecltrb}[8]{
\begin{tabular}{rcc}
 & masculine & feminine \\
sing. & #1 & #2 \\
& \textit{#3} & \textit{#4} \\
plur. & #5 & #6 \\
& \textit{#7} & \textit{#8}
\end{tabular}
}
\newcommand{\noundecltrc}[8]{
\begin{tabular}{rcc}
 & masculine & feminine \\
 sing. & \textbf{#1} & \textbf{#2} \\
& \textit{#3} & \textit{#4} \\
plur. & \textbf{#5} & \textbf{#6} \\
& \textit{#7} & \textit{#8}
\end{tabular}
}
\newcommand{\noundecltrd}[8]{
\begin{tabular}{rcc}
 & masculine & feminine \\
 \multirow{2}{*}{sing.} & \textbf{#1} & \textbf{#2} \\
& \textit{#3} & \textit{#4} \\
 \multirow{2}{*}{plur.} & \textbf{#5} & \textbf{#6} \\
& \textit{#7} & \textit{#8}
\end{tabular}
}

\newcommand{\noundefa}[4]{#1 (#3) (pl. #2): #4}
\newcommand{\noundefb}[4]{\textbf{#1} (#3) (pl. #2): \textit{#4}}
\newcommand{\noundefc}[5]{#1 \textbf{#2} (#4) (pl. #3): \textit{#5}}
\newcommand{\noundefd}[4]{#3 \textbf{#1} (pl. #2): \textit{#4}}

\begin{document}
 
\frontmatter

\maketitle

\tableofcontents

\mainmatter

\chapter{First castle}

\section{Phrase example}

\phrasea{Je m'appelle Pierre}{My name is Peter}\\
\phraseb{Je m'appelle Pierre}{My name is Peter}\\
\phrasec{Je m'appelle Pierre}{My name is Peter}\\
\phrased{Je m'appelle Pierre}{My name is Peter}

\section{Verb conjugation example}

\verbconja{Présent}{Boire}{bois}{bois}{boit}{buvons}{buvez}{boivent}
\verbconjb{Présent}{Boire}{bois}{bois}{boit}{buvons}{buvez}{boivent}
\verbconjc{Présent}{Boire}{bois}{bois}{boit}{buvons}{buvez}{boivent}

\begin{center}
\verbconja{Présent}{Boire}{bois}{bois}{boit}{buvons}{buvez}{boivent}
\quad
\verbconjd{Présent}{Boire}{bois}{bois}{boit}{buvons}{buvez}{boivent}
\end{center}

\section{Noun declension example}

\subsection{All four combinations}

\noundecla{étudiant}{étudiante}{étudiants}{étudiantes}

\begin{center}
\noundeclc{étudiant}{étudiante}{étudiants}{étudiantes}
\end{center}

\noundeclb{étudiant}{étudiante}{étudiants}{étudiantes}

\begin{center}
\noundecltra{celui-ci}{celle-ci}{this one}{this one}{ceux-ci}{celles-ci}{these ones}{these ones}
\quad
\noundecltrb{celui-là}{celle-là}{that one}{that one}{ceux-là}{celles-là}{those ones}{those ones}
\end{center}

\noundecltrc{celui-là}{celle-là}{that one}{that one}{ceux-là}{celles-là}{those ones}{those ones}

\noundecltrd{celui-là}{celle-là}{that one}{that one}{ceux-là}{celles-là}{those ones}{those ones}

\subsection{Noun definition}

\noundefa{voiture}{voitures}{f}{car}

\noundefb{voiture}{voitures}{f}{car}

\noundefc{une}{voiture}{voitures}{f}{car}

\noundefd{voiture}{voitures}{une}{car}




\end{document}

