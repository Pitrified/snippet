\documentclass[a4paper,11pt]{book}
\usepackage{import}

\usepackage[utf8]{inputenc}
\usepackage[T1]{fontenc}
\usepackage[italian]{babel}

\usepackage{lipsum}
\usepackage{enumitem} % more control over lists: stackoverflow.com/a/4974583
\usepackage{indentfirst} % YES indent after section/chapter heading

\usepackage{wrapfig}
\usepackage{adjustbox}

\newlength{\strutheight}
\settoheight{\strutheight}{\strut}

\usepackage{hyperref}	% enable hyperlinks to referenced elements
\hypersetup{
	colorlinks=true,
	linkcolor=cyan,
	citecolor=cyan,
    linkcolor=cyan,
    filecolor=magenta,
    urlcolor=blue,
}
\urlstyle{same}

\begin{document}

\chapter{Prove ingredienti}

\section{Crostata}

\subsection{Prima soluzione}

Proposta \url{https://tex.stackexchange.com/a/387324} qua.

\begin{adjustbox}{valign=T,raise=\strutheight,minipage={1.0\linewidth}}
    \begin{wrapfigure}{c}{0.55\linewidth}
        \begin{itemize}
            \item 300g farina
            \item 150g zucchero
            \item 100g burro
            \item 2 uova
            \item mezza bustina di lievito
            \item marmellata a piacere
        \end{itemize}
    \end{wrapfigure}% 
%
\lipsum[1]
\end{adjustbox}  

Per fare una crostata bla bla bla
\lipsum[2]

Succedono cose ssssstrane alla fine della box
 
\subsection{Seconda soluzione}

\begin{wrapfigure}{c}{0.55\linewidth}
    \begin{itemize}
        \item[--] 300g farina
        \item[--] 150g zucchero
        \item[--] 100g burro
        \item[--] 2 uova
        \item[--] mezza bustina di lievito
        \item[--] marmellata a piacere
    \end{itemize}
\end{wrapfigure}% boh c'era quando l'ho copiato

Mescolate zucchero e uova

Aggiungete il burro, eventualmente scaldato leggermente per semplicit\`a

Mescolate mezza bustina di lievito nella farina, ed aggiungetela mano a mano all'impasto, avendo cura di eliminare tutti i grumi

Per rendere la pasta pi\`u facile da maneggiare, la si pu\`o far riposare in frigo, coperta, per un quarto d'ora

Stendere la pasta sul fondo di una tortiera, creare un bordo con un rotolino di pasta;
Spalmate la marmellata;
Coprite la crostata con la tipica griglia, la pasta dovrebbe bastare per 6-8 rotoli

\subsection{Terza soluzione}

Testo per forzare il wrap
\begin{wrapfigure}{c}{0.55\linewidth}
    \begin{itemize}
        \item 300g farina
        \item 150g zucchero
        \item 100g burro
        \item 2 uova
        \item mezza bustina di lievito
        \item marmellata a piacere
    \end{itemize}
\end{wrapfigure}% boh c'era quando l'ho copiato
\begin{enumerate}
    \item Mescolate zucchero e uova
    \item Aggiungete il burro, eventualmente scaldato leggermente per semplicit\`a
    \item Mescolate mezza bustina di lievito nella farina, ed aggiungetela mano a mano all'impasto, avendo cura di eliminare tutti i grumi
    \item Per rendere la pasta pi\`u facile da maneggiare, la si pu\`o far riposare in frigo, coperta, per un quarto d'ora
    \item Stendere la pasta sul fondo di una tortiera, creare un bordo con un rotolino di pasta;
        Spalmate la marmellata;
        Coprite la crostata con la tipica griglia, la pasta dovrebbe bastare per 6-8 rotoli
\end{enumerate}

Succedono cose ancora pi\`u strane

\subsection{Quarta soluzione}

\textbf{Ingredienti:}

\emph{Per la pasta}
% \begin{itemize}[noitemsep,parsep=0pt,partopsep=0pt,topsep=0pt]
% \begin{itemize}[noitemsep]
\begin{itemize}[itemsep=1pt]
% \begin{itemize}
    \item[--] 300g farina
    \item[--] 150g zucchero
    \item[--] 100g burro
    \item[--] 2 uova
    \item[--] mezza bustina di lievito
    \item[--] marmellata a piacere
\end{itemize}

\emph{Per la marmellata}
\begin{itemize}[itemsep=1pt]
    \item[--] 1kg albicocche
    \item[--] 350g zucchero
    \item[--] una bustina di pecina
\end{itemize}

\textbf{Procedimento:}

Mescolate zucchero e uova;
Aggiungete il burro, eventualmente scaldato leggermente per semplicit\`a

Mescolate mezza bustina di lievito nella farina, ed aggiungetela mano a mano all'impasto, avendo cura di eliminare tutti i grumi

Per rendere la pasta pi\`u facile da maneggiare, la si pu\`o far riposare in frigo, coperta, per un quarto d'ora

Stendere la pasta sul fondo di una tortiera, creare un bordo con un rotolino di pasta;
Spalmate la marmellata;
Coprite la crostata con la tipica griglia, la pasta dovrebbe bastare per 6-8 rotoli
\end{document}
